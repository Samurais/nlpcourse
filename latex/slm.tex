\documentclass[ignorenonframetext]{beamer}

\newcommand{\vocab}{\mathcal{V}}
\newcommand{\corpus}{\mathcal{C}}
\newcommand{\pml}{p_{\textsc{ml}}}
\newcommand{\sos}{$<$s$>$}
\newcommand{\eos}{$<$/s$>$}

\title{Learning to map strings to probabilities}
\subtitle{Comp 542 Natural Language Processing}
\author{Deniz Yuret}

\hypersetup{colorlinks,urlcolor=red}

\begin{document}

\begin{frame}
\maketitle
\end{frame}


TODO:
- add section on smoothing methods and why they work?  proof of kneser ney?
- can we do a bayesian one?  mackay-peto, teh pitman yor?
- is there a conditional one?  rosenfeld has maxent, introduce?
- is there a discriminative one?  does that make sense?
- significance testing more detail
- prepare quiz and homework

LEARNING OUTCOMES:

- Generative models
- Maximum likelihood
- Overfitting and regularization (applying a prior) by smoothing (discounting)
- Entropy and perplexity (specific to slm)
- Uncertainty in experiments


Statistical Language Modeling

Statistical language models assign probabilities to
strings\footnote{Strings are sequences of tokens where each token
  comes from a finite vocabulary.  We will usually consider strings
  where each token is a word, number, or a punctuation mark.  Other
  examples of strings include DNA sequences, stock market moves,
  musical notes, etc.}  in a language\footnote{This is as opposed to
  grammars which specify what strings are part of a language and what
  strings are not.  So we can view statistical language models as soft
  grammars.}.  In this section we will cover n-gram language models
which specify a probability distribution over strings by specifying
which tokens can follow a given $n-1$ token prefix with what
probability.  We will also introduce the concepts of generative
models, maximum likelihood, entropy/perplexity, and significance
testing.

Why do we need them?

Noisy channel models: X --> Y
X is a string we'd like to know.
Y is what we get (OCR, Speech, Translation)
P(X|Y) ~ P(Y|X) P(X)
The P(X) comes from the LM


Generative Models

In probability and statistics, a generative model is a model for
randomly generating observable data, typically given some hidden
parameters.  An n-gram language model (also confusingly known as a
Markov chain of order n-1) is a model where the choice of each token
depends on the past n-1 tokens.

It is sometimes convenient to think of a generative model as a
probabilistic program that can generate artificial data that looks
like our observations.  We will call such programs generative
procedures.  Here is the generative procedure for an n-gram language
model:

w[0] = <s>  // a special symbol denoting the start of the string
for (i = 0; w[i] != </s>; i++) {
  let p = a probability distribution over V+</s> conditional on the
  last n-1 tokens up to and including w[i] (if there aren't n-1 tokens
  yet shorten the context accordingly)
  pick w[i+1] from V+</s> with probability p
}

The procedure deterministically starts every string with the token
<s>.  It then starts generating tokens from the set of tokens in the
vocabulary and possibly the special end-of-string token </s>.  It uses
categorical distributions\footnote{A categorical distribution is a
  generalization of Bernoulli distribution.  Bernoulli distribution
  specifies the probabilities of two possible outcomes, categorical
  distribution specifies the probabilities of |V| possible outcomes.}
over V+</s> that depend on the last n-1 tokens generated.

Note that the procedure uses a different probability distribution for
each of the |V|^(n-1) different prefixes.  Each of these distributions
specify |V+</s>| probabilities.  These all must be supplied to the
program as parameters before it can work.  Once we have these O(V^N)
parameters we can answer questions such as:

- What is the probability of this program generating a certain string
- What tokens are most likely to follow a given string
- What are the most likely alternatives in the middle of a string
- What tokens are most likely to precede a given string

Review product rule, sum rule:

p(a or b) = p(a) + p(b) if a, b disjoint (-p(a and b) otherwise)
p(a and b) = p(a) p(b | a) = p(b) p(a | b)  always
// cox has product rule and negation rule, no sum rule?



Maximum Likelihood

If we have parameters we can assign probabilities.  How do we get
parameters?  From training data.

If a context c is observed n times and it is followed by token w k
times, what is the maximum likelihood estimate for p(w|c)?
// likelihood is a function of q!

Find q such that p(data|q) is maximized.

data = sequence of k w's and n-k non-w's

p(data|q) = q^k (1-q)^(n-k)

logp(data|q) = k log(q) + (n-k) log(1-q)

dlogp/dq = k / q - (n-k) / (1-q) = 0

k (1-q)  = (n-k) q
k - kq = nq - kq
q = k/n

In general the ML estimate gives probabilities proportional to
observation frequencies.

Q: Calculate the ML estimate for gaussian mean for fixed variance.



Evaluation

So given parameters we can assign probabilities to every string using
the generative model.

Given training data we can generate parameters using maximum
likelihood.

ML is not the only way to generate parameters.  There are other
methods.  How do we compare them?  

We need an unseen test set and an evaluation function.

We compare log likelihood on the test set:

eval1 = sum_i log pi

Related measures:

entropy H(p) = E_p(-logp) = -sum_w pw log pw 
// number of bits in ideal compression
// arithmetic average of log(1/p)

cross-entropy H(p,q) = E_p(-logq) = -sum_w pw log qw  
// what would happen if we used q instead of perfect p

KL distance D(p,q) = sum_w pw log pw - sum_w pw log qw  
// cost of using q instead of p. = H(p,q) - H(p)

perplexity Perp(p) = 2^H(p)
// give examples of uniform distribution
// geometric average of 1/p

Q: is it possible to beat ML on the training set?


Significance testing

If we have more than one way of estimating parameters, assigning
probabilities how can we be sure that one is better than the other?

Answer: try lots of times (cross validation)
compute standard error

Reasons for variation: 
Given test set and training set it is all deterministic (may not be)
Training set randomness (same algorithm gives different models)
Test set randomness (same model gives different test set results)

Simple example: standard error of a bernoulli parameter
We estimate q = k/n, how good is this estimate?
Related question: if real parameter were p, and we tried this lots of
times, how spread out would q be?

q ~ sum of n bernoulli variables
p(1)=p, p(0)=1-p, E(x)=p, E(x^2)=p, Var(x)=E(x^2) - E(x)^2 = p-p*p 
E(q) = p
Var(nq) = n Var(x) = np(1-p)
Var(Cx) = C^2 Var(x)
Var(q) = p(1-p)/n < 1/4n
Std(q) < 1/(2 sqrt(n))

Example: p=1/2
n=100 => Std(q) < 1/20 = 5\%  // i.e [45%,55%] is the one stderr window
n=10000 => Std(q) < 1/200 = .5\% // i.e. [49.5%,50.5%]

Question: if you have two values and their standard errors, when can
you say one is better than the other?

Read student's t-test from wikipedia.

Can't we do back of the envelope?

Need to define paired / unpaired, equal / unequal variance, p-values
etc.

p-value is the area under the curve of the test statistic for values
that are more extreme than the observed.

Recommended method:

Run 10-fold cross validation.
Test both algorithms on same folds.
Plot standard error bars.
We have 10 paired values that can be used in dependent t-test for
paired samples.

We have two things to compute stderr with: test set size and
cross-validation fold differences.  Test set size is probably not
easily applicable for anything except correct/incorrect?  Fold
differences are always ok.

David Barber has a Bayesian test:
http://citeseerx.ist.psu.edu/viewdoc/download?doi=10.1.1.11.293&rep=rep1&type=pdf

StackExchange recommends McNemar test for a single run:
http://stats.stackexchange.com/questions/26271/comparing-two-classifier-accuracy-results-for-statistical-significance-with-t-te
http://en.wikipedia.org/wiki/McNemar%27s_test

Chapter 5.3 in manning and schutze covers significance testing.



Overfitting and Regularization

ML models overfit, they give too much probability to data they have
seen and too little to data they have not seen.  In case of SLM, if we
have observed 2 distinct words following an n-1-gram all the other
words get zero probability, which is unreasonable and will do poorly
on the test set.

Q: what happens to log-likelihood and perplexity when an unseen word
appears with the ML model?


** Backoff vs Interpolated

       N-gram models try to estimate the probability of a word z in the context of the  previous  n-1  words
       (a_), i.e., Pr(z|a_).  We will denote this conditional probability using p(a_z) for convenience.  One
       way to estimate p(a_z) is to look at the number of times word z has followed the previous  n-1  words
       (a_):

       (1)  p(a_z) = c(a_z)/c(a_)

\begin{frame}\frametitle{Maximum Likelihood}
\begin{equation}
\pml(c|ab) = \frac{C(abc)}{C(ab)}
\end{equation}
\end{frame}

       This  is  known  as  the  maximum likelihood (ML) estimate.  Unfortunately it does not work very well
       because it assigns zero probability to N-grams that have not been observed in the training data.   To
       avoid  the zero probabilities, we take some probability mass from the observed N-grams and distribute
       it to unobserved N-grams.  Such redistribution is known as smoothing or discounting.


Discount methods: instead of p_ML = Cw/C use:
- Additive: (Cw+A)/(C+VA)
- Absolute discounting: (Cw-A)/C
- Jelinek-Mercer: \lambda * (Cw/C)
- MacKay-Peto (Dirichlet): Cw/(C+A)
- Good-Turing: (Cw+1)/C * V(Cw+1)/V(Cw)
- Kneser-Ney: (Cw-A)/C but with modified lower order

\begin{frame}\frametitle{Discount (Smoothing) Methods}
\begin{eqnarray*}
p(c|ab) & \propto & C(abc) + A \\
p(c|ab) & \propto & C(abc) - D \\
p(c|ab) & \propto & \lambda C(abc) \\
p(c|ab) & \propto & C(abc) / (C(ab) + A) \\
p(c|ab) & \propto & (C(abc)+1)\vocab_{C+1}/\vocab_{C}
\end{eqnarray*}
\end{frame}

       Most existing smoothing algorithms can be described by the following equation:

       (2)  p(a_z) = (c(a_z) > 0) ? f(a_z) : bow(a_) p(_z)   ; backoff

       (4)  p(a_z) = g(a_z) + bow(a_) p(_z)   ; interpolated

       If the N-gram a_z has been observed in the training data, we use the distribution f(a_z).   Typically
       f(a_z)  is  discounted to be less than the ML estimate so we have some leftover probability for the z
       words unseen in the context (a_).  Different algorithms mainly differ on how  they  discount  the  ML
       estimate to get f(a_z).

       If  the  N-gram  a_z  has not been observed in the training data, we use the lower order distribution
       p(_z).  If the context has never been observed (c(a_) = 0), we can use the lower  order  distribution
       directly  (bow(a_) = 1).  Otherwise we need to compute a backoff weight (bow) to make sure probabili-
       ties are normalized:

            Sum_z p(a_z) = 1


** Show normalization calculation for some of them:

Normalization for Additive (there are no zero counts, so no backoff):
    
   p(a_z)  = (c(a_z) + A) / (c(a_) + AV)  ; additive smoothing
   Sum_Z p(a_z) = Sum_Z c(a_z) / (c(a_) + AV) + Sum_Z A / (c(a_) + AV)
                = 1

\begin{frame}\frametitle{Additive Smoothing}
\begin{eqnarray*}
p(c|ab) &=& \frac{C(abc) + A}{C(ab) + A\vocab} \\
\sum_{c\in\vocab} p(c|ab) &=& \sum_{c\in\vocab} \frac{C(abc)}{C(ab) +
  A\vocab} + \sum_{c\in\vocab} \frac{A}{C(ab) + A\vocab} \\
  &=& \frac{C(ab)}{C(ab) + A\vocab} + \frac{A\vocab}{C(ab) + A\vocab}\\
  &=& 1
\end{eqnarray*}
\end{frame}


Normalization for Jelinek Mercer:

   p(a_z)  = g(a_z) + bow(a_) p(_z)   ; Eqn.4 - interpolated model
   g(a_z)  = D c(a_z) / c(a_)         ; Jelinek Mercer
   bow(a_) = 1 - Sum_Z1 g(a_z)
           = 1 - Sum_Z1 D c(a_z) / c(a_)
           = 1 - D

\begin{frame}\frametitle{Jelinek-Mercer (Deleted Interpolation)}
\begin{eqnarray*}
p(c|ab) &=& (1-\gamma(ab)) \frac{C(abc)}{C(ab)} + \gamma(ab) p(c|b) \\
\sum_{c\in\vocab} p(c|ab) &=& (1-\gamma(ab)) \sum_{c\in\vocab}
\frac{C(abc)}{C(ab)} + \gamma(ab) \sum_{c\in\vocab} p(c|b) \\
&=& (1-\gamma(ab)) (1) + \gamma(ab) (1) \\
&=& 1
\end{eqnarray*}
\end{frame}

BOW for interpolated absolute discounting:

   p(a_z)  = g(a_z) + bow(a_) p(_z)   ; Eqn.4 - interpolated model
   g(a_z)  = max(0, c(a_z) - D) / c(a_) ; absolute discounting
   bow(a_) = 1 - Sum_Z1 g(a_z)        ; Eqn.5
           = 1 - Sum_Z1 (c(a_z) - D) / c(a_)
           = 1 - Sum_Z1 c(a_z)/c(a_) + Sum_Z1 D/c(a_)
           = D n(a_*) / c(a_)

              The suggested discount factor is:

                   D = n1 / (n1 + 2*n2)

\begin{frame}\frametitle{Absolute Discounting}
\begin{eqnarray*}
f(c|ab) &=& \frac{\max(0, C(abc) - D)}{C(ab)} \\
p(c|ab) &=& f(c|ab) + \gamma(ab) p(c|b) \\
\sum_{c\in\vocab} p(c|ab) &=& \frac{\sum_{c\in\vocab} \max(0,
  C(abc)-D)}{C(ab)} + \gamma(ab) \sum_{c\in\vocab} p(c|b) \\
1 &=& \frac{\sum_{c\in\vocab_{+}} C(abc)-D}{C(ab)} + \gamma(ab)
(1) \\ 
\gamma(ab) &=& 1 - \frac{\sum_{c\in\vocab_{+}} C(abc)}{C(ab)} +
\frac{\sum_{c\in\vocab_{+}} D}{C(ab)} \\
&=& \frac{|\vocab_{+}| D}{C(ab)}
\end{eqnarray*}
\end{frame}

BOW for Interpolated Dirichlet:

   p(a_z)  = g(a_z) + bow(a_) p(_z)   ; Eqn.4 - interpolated model
   g(a_z)  = c(a_z) / (c(a_) + A)         ; Dirichlet
   bow(a_) = 1 - Sum_Z1 g(a_z)
           = 1 - Sum_Z1 c(a_z) / (c(a_) + A)
           = 1 - c(a_) / (c(a_) + A)
           = A / (c(a_) + A)


\begin{frame}\frametitle{Dirichlet?}
\begin{eqnarray*}
f(c|ab) &=& \frac{C(abc)}{C(ab) + A} \\
p(c|ab) &=& f(c|ab) + \gamma(ab) p(c|b) \\
\sum_{c\in\vocab} p(c|ab) &=& \frac{\sum_{c\in\vocab} C(abc)}{C(ab) + A} + \gamma(ab) \sum_{c\in\vocab} p(c|b) \\
1 &=& \frac{C(ab)}{C(ab) + A} + \gamma(ab) (1) \\ 
\gamma(ab) &=& 1 - \frac{C(ab)}{C(ab) + A} \\
&=& \frac{A}{C(ab) + A} \\
\end{eqnarray*}
\end{frame}

       General Normalization:

       Let Z be the set of all words in the vocabulary, Z0 be the set of all words with c(a_z) = 0,  and  Z1
       be the set of all words with c(a_z) > 0.  Given f(a_z), bow(a_) can be determined as follows:

       (3)  Sum_Z  p(a_z) = 1
            Sum_Z1 f(a_z) + Sum_Z0 bow(a_) p(_z) = 1
            bow(a_) = (1 - Sum_Z1 f(a_z)) / Sum_Z0 p(_z)
                    = (1 - Sum_Z1 f(a_z)) / (1 - Sum_Z1 p(_z))
                    = (1 - Sum_Z1 f(a_z)) / (1 - Sum_Z1 f(_z))

\begin{frame}\frametitle{Back-off vs. Interpolated}
\begin{eqnarray*}
p(c|ab) &=& (C(abc) > 0) ? f(c|ab) : \gamma(ab) p(c|b) \\
p(c|ab) &=& f(c|ab) + \gamma(ab) p(c|b)
\end{eqnarray*}
\end{frame}

\begin{frame}\frametitle{Normalization in back-off models}
\begin{eqnarray*}
p(c|ab) &=& (C(abc) > 0) ? f(c|ab) : \gamma(ab) p(c|b) \\
\sum_{c\in\vocab} p(c|ab) &=& \sum_{c\in\vocab_{+}} f(c|ab) +
\gamma(ab) \sum_{c\in\vocab_{0}} p(c|b) \\
\gamma(ab) &=& \frac{1 - \sum_{c\in\vocab_{+}}
  f(c|ab)}{\sum_{c\in\vocab_{0}} p(c|b)} \\
&=& \frac{1 - \sum_{c\in\vocab_{+}} f(c|ab)}{1 - \sum_{c\in\vocab_{+}}
  p(c|b)} \\
&=& \frac{1 - \sum_{c\in\vocab_{+}} f(c|ab)}{1 - \sum_{c\in\vocab_{+}}
  f(c|b)} \\
\end{eqnarray*}
\end{frame}

       Smoothing  is  generally  done in one of two ways.  The backoff models compute p(a_z) based on the N-
       gram counts c(a_z) when c(a_z) > 0, and only consider lower order  counts  c(_z)  when  c(a_z)  =  0.
       Interpolated  models  take  lower order counts into account when c(a_z) > 0 as well.  A common way to
       express an interpolated model is:

       (4)  p(a_z) = g(a_z) + bow(a_) p(_z)

       Where g(a_z) = 0 when c(a_z) = 0 and it is discounted to be less than the ML estimate when c(a_z) > 0
       to  reserve some probability mass for the unseen z words.  Given g(a_z), bow(a_) can be determined as
       follows:

       (5)  Sum_Z  p(a_z) = 1
            Sum_Z1 g(a_z) + Sum_Z bow(a_) p(_z) = 1
            bow(a_) = 1 - Sum_Z1 g(a_z)

\begin{frame}\frametitle{Normalization in interpolated models}
\begin{eqnarray*}
p(c|ab) &=& f(c|ab) + \gamma(ab) p(c|b) \\
\sum_{c\in\vocab} p(c|ab) &=& \sum_{c\in\vocab_{+}} f(c|ab) +
\gamma(ab) \sum_{c\in\vocab} p(c|b) \\
\gamma(ab) &=& 1 - \sum_{c\in\vocab_{+}} f(c|ab)
\end{eqnarray*}
\end{frame}

** ARPA Format:

       An interpolated model can also be expressed in the form of equation (2), which is the way it is  rep-
       resented in the ARPA format model files in SRILM:

       (6)  f(a_z) = g(a_z) + bow(a_) p(_z)
            p(a_z) = (c(a_z) > 0) ? f(a_z) : bow(a_) p(_z)

       Most  algorithms  in  SRILM  have  both backoff and interpolated versions.  Empirically, interpolated
       algorithms usually do better than the backoff ones, and Kneser-Ney does better than others.

\begin{frame}[fragile]\frametitle{ARPA format}
\scriptsize
\begin{verbatim}
\data\
ngram 1=49208
ngram 2=402095
ngram 3=105292

\1-grams:
...
-5.242319	approves	-0.1590842
-5.388447	approving	-0.2031582
-5.610296	approximate	-0.1472198
...
\2-grams:
...
-2.774657	approval but
-0.8375884	approval by	-0.1785298
-2.774657	approval came
...
\3-grams:
...
-1.121324	approval , the
-0.806105	approval , which
-1.121324	approval , would
\end{verbatim}
\end{frame}

\begin{frame}\frametitle{Interpolated models in ARPA format}
\begin{eqnarray*}
g(c|ab) &=& f(c|ab) + \gamma(ab) p(c|b) \\
p(c|ab) &=& (C(abc) > 0) ? g(c|ab) : \gamma(ab) p(c|b)
\end{eqnarray*}
\end{frame}

** Motivation for discount methods:

All arithmetic operations: add (additive), subtract (abs discount),
multiply (jelinek, gt) nominator, add to denominator (dirichlet).  We
can use absolute discount as a regularization method with the single
discount parameter controlling the amount of regularization.  Or maybe
Jelinek Mercer or additive is a better example.  Dirichlet to
illustrate probabilistic.

** Motivation for Good-Turing

Good-Turing proved that the total prob for future unseen words is the
current frequency of one-counts.

In general they suggest that the total prob for n-count words should
be set to the current frequency of (n+1)-count words.

This can be accomplished by modifying the counts:

Cw' = (Cw+1) V(Cw+1)/V(Cw)  where V(Cw) is the number of words with Cw-count

Notation: C is the corpus (or corpus count), V is the vocabulary (or
vocab count).  Cw is the count of w in corpus.  Vn is the number of
words with count=n.

Goodman example:

Imagine you are fishing

There are 8 species: carp, perch, whitefish, trout, salmon, eel,
catfish, bass

You have caught 10 carp, 3 perch, 2 whitefish, 1 trout, 1 salmon, 1
eel

How likely is it that next species is new (I.e. catfish or bass) 
3/18

Assuming so, how likely is it that next species is trout?  
Less than 1/18


In this example: 
C=18: sum of Cc=10, Cp=3, Cw=2, Ct=1, Cs=1, Ce=1, Ccat=0, Cb=0.
V=8: sum of V0=2, V1=3, V2=1, V3=1, V10=1.

So all zero counts should share V1/C equally, they get (1/C)(V1/V0)
=(1/18)(3/2)=1/12 => total=2/12

Each one count word gets: (2/C)(V2/V1), total=(2/C)V2
This is what the total ML prob was for two counts!
=(2/18)(1/3)=1/27 => total=1/9

Each two count word gets: (3/C)(V3/V2), total=(3/C)V3
This is what the total ML prob was for three counts!
=(3/18)(1/1)=1/6 => total=1/6

Each three count word gets: (4/C)(V4/V3), total=(4/C)V4
Problem: V4=0, ooops.
So counts need to be smoothed or need a cut-off.  See Chen&Goodman,
Katz smoothing for details.

\begin{frame}\frametitle{Good-Turing}
\begin{eqnarray*}
r_{GT} &=& (r+1)\frac{\vocab_{r+1}}{\vocab_{r}} \\
|\corpus| &=& \sum_{r=0}^\infty r\vocab_r \\
&=?& \sum_{r=0}^\infty r_{GT}\vocab_r \\
&=& \sum_{r=0}^\infty (r+1)\vocab_{r+1} \\
&=& \sum_{r=1}^\infty r \vocab_r  \mbox{  this works if  } \forall r : \vocab_r > 0
\end{eqnarray*}
\end{frame}

\begin{frame}\frametitle{Goodman example}
\begin{eqnarray*}
\vocab &=& \{ \mbox{carp, perch, whitefish, trout, salmon, eel,
  catfish, bass} \} \\
\corpus &=& [ \mbox{carp x 10, perch x 3, whitefish x 2, trout, salmon, eel} ] \\
|\corpus| &=& 18, |\vocab| = 8, \vocab_0=2, \vocab_1=3, \vocab_2=1, \vocab_3=1,
\vocab_{10}=1 \\
\end{eqnarray*}
\begin{eqnarray*}
p_{ML}(\mbox{trout}) &=& 1/18 \\
p_{GT}(\mbox{trout}) &=& \frac{2 \vocab_2 / \vocab_1}{|\corpus|} = 1/27 \\
p_{ML}(\mbox{bass}) &=& 0 \\
p_{GT}(\mbox{bass}) &=& \frac{ \vocab_1 / \vocab_0}{|\corpus|} = 1/12 \\
p_{ML}(\mbox{perch}) &=& 1/6 \\
p_{GT}(\mbox{perch}) &=& 0!!! \\
\end{eqnarray*}

\end{frame}

\begin{frame}\frametitle{Katz}
\begin{columns}[t]
\column{.5\textwidth}
\begin{eqnarray*}
r^* &=& r \mbox{ , } \forall r>k \\
C &=& \sum_{r=0}^{\infty} r\vocab_r = \sum_{r=0}^{\infty} r^*\vocab_r \\
r-r^* &=& \mu (r-r_{GT}) \\
0^* \vocab_0 &=& \vocab_1 = \sum_{r=1}^k (r-r^*)\vocab_r \\
&=& \sum_{r=1}^k \mu (r-r_{GT})\vocab_r \\
&=& \mu \sum_{r=1}^k r\vocab_r - (r+1)\vocab_{r+1} \\
&=& \mu (\vocab_1 - (k+1)\vocab_{k+1}) \\
\end{eqnarray*}
\column{.5\textwidth}
\begin{eqnarray*}
\mu &=& \frac{\vocab_1}{\vocab_1 - (k+1)\vocab_{k+1}} \\
r^* &=& r - \mu r + \mu r_{GT} \\
d_r &=& r^*/r \\
&=& 1 - \mu + \mu r_{GT}/r \\
&=& \frac{r_{GT}/r - (k+1)\vocab_{k+1}/\vocab_1}{1 - (k+1)\vocab_{k+1}/\vocab_1}
\end{eqnarray*}
\end{columns}
\end{frame}

** Motivation for Kneser-Ney

       (2)  p(a_z) = (c(a_z) > 0) ? f(a_z) : bow(a_) p(_z)   ; backoff
       (4)  p(a_z) = g(a_z) + bow(a_) p(_z)   ; interpolate

We use p(_z) only when c(a_z)=0.  But we calculate it like we use it
all the time.  How can we use the fact that c(a_z) was 0?

Bigram example ``San Francisco''.  Say Francisco only appears after San.

c(sf) = large
c(xf) = 0 for x != s

p(xf) should be very close to 0.

p(xf) = bow(x_) p(f)

p(f) is high, which will make p(xf) high!

We should have estimated p(f) as low since it is only going to be used
if not after following san.

So set p(f) to V(*f), i.e. number of different words it follows,
rather than C(*f)=C(f), i.e. number of times it was observed.

To derive the formula:
(i) use interpolated form (easier and better)
       (4)  p(a_z) = g(a_z) + bow(a_) p(_z)   ; interpolate

(ii) satisfy the marginal constraints:

  sum_a p(ab) = Cb / C = p_ML(b)   ;; where p is the KN estimate

= sum_a p(b|a) p(a)   ;; use Ca/C for p(a)

= sum_a p(b|a) Ca/C = Cb/C

Cb = sum_a Ca p(b|a) 

   = sum_a Ca (max(0, Cab - D)/Ca + bow(a) p(b))

bow(a) = V(a_) D / Ca

Cb = sum_a Ca (max(0, Cab - D)/Ca + Va_ (D/Ca) p(b))
   = sum_a max(0, Cab - D) + sum_a Va_ D p(b)

p(b) = (Cb - sum_a max(0, Cab - D)) / sum_a Va_ D

sum_a max(0, Cab - D) = Cb - V_b D

p(b) = V_b D / sum_a Va_ D = V_b / sum_a Va_ = V_b / V

\begin{frame}\frametitle{Kneser-Ney}
\begin{eqnarray*}
\pml(b) &=& C(b)/C = \sum_a p(ab) \\
&=& \sum_a p(a) p(b|a) \\
&\approx& \sum_a \frac{C(a)}{C} \left( \frac{\max(C(ab) - D, 0)}{C(a)}
+ \frac{p(b) \vocab_{a*} D}{C(a)} \right) \\
C(b) &=& \sum_{a\in\vocab_{*b}} C(ab) - \sum_{a\in\vocab_{*b}} D +
p(b) D \sum_{a\in\vocab} \vocab_{a*} \\
&=& C(b) - \vocab_{*b} D + p(b) D \vocab_{**} \\
p(b) &=& \vocab_{*b} / \vocab_{**} \\
\end{eqnarray*}
\end{frame}

** BOW calculation for Kneser-Ney

-kndiscount and -ukndiscount
       Kneser-Ney discounting.  This is similar to absolute discounting in that the discounted proba-
       bility is computed by subtracting a constant D from the N-gram count.  The options -kndiscount
       and -ukndiscount differ as to how this constant is computed.
       The  main idea of Kneser-Ney is to use a modified probability estimate for lower order N-grams
       used for backoff.  Specifically, the modified probability for a lower order N-gram is taken to
       be proportional to the number of unique words that precede it in the training data.  With dis-
       counting and normalization we get:

            f(a_z) = (c(a_z) - D0) / c(a_)     ;; for highest order N-grams
            f(_z)  = (n(*_z) - D1) / n(*_*)    ;; for lower order N-grams

       where the n(*_z) notation represents the number of unique N-grams that match a  given  pattern
       with  (*) used as a wildcard for a single word.  D0 and D1 represent two different discounting
       constants, as each N-gram order uses a different discounting constant.  The  resulting  condi-
       tional probability and the backoff weight is calculated as given in equations (2) and (3):

            p(a_z)  = (c(a_z) > 0) ? f(a_z) : bow(a_) p(_z)     ; Eqn.2
            bow(a_) = (1 - Sum_Z1 f(a_z)) / (1 - Sum_Z1 f(_z))  ; Eqn.3

       The  option  -interpolate  is  used  to  create  the  interpolated versions of -kndiscount and
       -ukndiscount.  In this case we have:

            p(a_z) = g(a_z) + bow(a_) p(_z)  ; Eqn.4

       Let Z1 be the set {z: c(a_z) > 0}.  For highest order N-grams we have:

            g(a_z)  = max(0, c(a_z) - D) / c(a_)
            bow(a_) = 1 - Sum_Z1 g(a_z)
                    = 1 - Sum_Z1 c(a_z) / c(a_) + Sum_Z1 D / c(a_)
                    = D n(a_*) / c(a_)

       Let Z2 be the set {z: n(*_z) > 0}.  For lower order N-grams we have:

            g(_z)  = max(0, n(*_z) - D) / n(*_*)
            bow(_) = 1 - Sum_Z2 g(_z)
                   = 1 - Sum_Z2 n(*_z) / n(*_*) + Sum_Z2 D / n(*_*)
                   = D n(_*) / n(*_*)

       The original Kneser-Ney discounting (-ukndiscount) uses one discounting constant for  each  N-
       gram order.  These constants are estimated as

            D = n1 / (n1 + 2*n2)

       where n1 and n2 are the total number of N-grams with exactly one and two counts, respectively.
       Chen and Goodman's modified Kneser-Ney discounting (-kndiscount) uses three  discounting  con-
       stants  for  each  N-gram order, one for one-count N-grams, one for two-count N-grams, and one
       for three-plus-count N-grams:

            Y   = n1/(n1+2*n2)
            D1  = 1 - 2Y(n2/n1)
            D2  = 2 - 3Y(n3/n2)
            D3+ = 3 - 4Y(n4/n3)


** Motivation for Mackay-Peto

Next lecture, possibly with Teh...

MacKay explains Jelinek, Teh explains Kneser-Ney the Bayesian way?

%% \begin{frame}\frametitle{Bernoulli, Binomial, Beta}
%% \begin{itemize}

%% \end{itemize}
%% \end{frame}

%% \begin{frame}\frametitle{Laplace - Rule of Succession}

%% \end{frame}

%% \begin{frame}\frametitle{Categorical, Multinomial, Dirichlet}

%% \end{frame}

%% \begin{frame}\frametitle{Laplace and Additive Smoothing}

%% \end{frame}

%% \begin{frame}\frametitle{MacKay-Peto and Jelinek-Mercer}

%% \end{frame}

%% \begin{frame}\frametitle{Dirichlet, Pitman-Yor Processes}

%% \end{frame}

%% \begin{frame}\frametitle{Teh and Kneser-Ney}

%% \end{frame}

\end{document}
