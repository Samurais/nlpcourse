\documentclass[ignorenonframetext]{beamer}

\newcommand{\vocab}{\mathcal{V}}
\newcommand{\corpus}{\mathcal{C}}
\newcommand{\pml}{p_{\textsc{ml}}}
\newcommand{\sos}{$<$s$>$}
\newcommand{\eos}{$<$/s$>$}

\title{Interlude on Probabilistic Inference}
\subtitle{Comp 542 Natural Language Processing}
\author{Deniz Yuret}

\hypersetup{colorlinks,urlcolor=red}

\begin{document}

\begin{frame}
\maketitle
\end{frame}

** Motivation for Mackay-Peto

Next lecture, possibly with Teh...

MacKay explains Jelinek, Teh explains Kneser-Ney the Bayesian way?

\begin{frame}\frametitle{Probabilistic Inference (\href{http://www2.denizyuret.com/ref/mackay/lang4.pdf}{Mackay and Peto, 1995})}
\begin{itemize}
\item Make all assumptions explicit: A model $\mathcal{H}$ specifies
  the parameters $Q$, their prior probability distribution
  $P(Q|\mathcal{H})$, and the way that the probability of the data $D$
  depends on the parameters $P(D|Q,\mathcal{H})$.
\item Infer parameters $Q$ given data $D$:\footnote{The normalizing constant (evidence) is given by integrating the
numerator:\[ P(D|\mathcal{H}) = \int P(D|Q,\mathcal{H})
P(Q|\mathcal{H}) dQ
\]
}
$\color{red}\textsl{posterior}=\frac{\textsl{likelihood}\times
  \textsl{prior}}{\textsl{evidence}}$
\[
P(Q|D,\mathcal{H}) = \frac{P(D|Q,\mathcal{H}) P(Q|\mathcal{H})}{P(D|\mathcal{H})}
\]
\item Predict future data $D'$ given past $D$ {\small(predictive distribution)}:
\[
P(D'|D,\mathcal{H}) = \int P(D'|Q,\mathcal{H}) P(Q|D,\mathcal{H}) dQ
\]
\end{itemize}
\end{frame}

\begin{frame}\frametitle{Cox Axioms (from \href{http://www.inference.phy.cam.ac.uk/itila}{MacKay, 2003})}\small

\begin{description}
%% \item[Notation\puncspace] Let `the degree of belief in proposition
%%   $x$'
%%   be denoted by $B(x)$. The negation of $x$
%%   ({\sc not}-$x$) is written $\overline{x}$.
%%   The degree of belief in a conditional proposition,
%%   %for example, the belief in
%%   `$x$, assuming proposition $y$ to be true',
%%   is represented by $B(x \given y)$.
\item[Axiom 1\puncspace] Degrees of belief can be ordered;
  if $B(x)$ is `greater' than $B(y)$, and 
  $B(y)$ is `greater' than $B(z)$, then
   $B(x)$ is `greater' than $B(z)$.
  ({\sf Consequence:} beliefs can be mapped onto real numbers.)
\item[Axiom 2\puncspace]  The degree of belief in a proposition $x$
  and its negation  $\overline{x}$ are related. There is a
  function $f$ such that
\[
B(x) = f[ B( \overline{x} ) ] .
\]
\item[Axiom 3\puncspace]   The degree of belief in a conjunction  of
  propositions
  $x,y$ ($x\,\mbox{\sc and}\,y$)
  is related to 
 the degree of belief in the conditional proposition $x | y$
 and  the degree of belief in the proposition $y$.
 There is a function  $g$ such that
\[
B(x,y) = g \left[ B(x | y)  , B(y) \right] .
\]
\end{description}

If a set of beliefs satisfy these
 axioms then they can be mapped onto probabilities satisfying
 $P(\mbox{\sc false}) = 0$,
 $P(\mbox{\sc true}) = 1$,
 $0 \leq P(x) \leq  1$, and 
 the rules of probability: $
 P(x) = 1 - P( \overline{x} ) \mbox{ and }
 P(x,y) =  P(x | y)  P(y).
$
\end{frame}

\begin{frame}\frametitle{\textcolor{red}{Bernoulli}, Binomial, Beta}
The Bernoulli distribution models two possible outcomes (k=0 or k=1)
with parameter $p\in [0,1]$:\[
\Pr(X=k) = \begin{cases} p & \text{if }k=1, \\[6pt] 1-p & \text {if
  }k=0.\end{cases}
\]
Here is an alternative way to express the Bernoulli pmf:\[
\Pr(X=k) = p^k (1-p)^{1-k}\!\quad \text{for }k\in\{0,1\}.
\]
A Bernoulli random variable has expected value \[
\mbox{E}(X)=p
\] and variance \[
\mbox{Var}(X)=p(1-p)
\]
\end{frame}

\begin{frame}\frametitle{Bernoulli, \textcolor{red}{Binomial}, Beta}
The Binomial distribution models the number of successes (nonzero outcomes)
in $n$ independent Bernoulli trials with parameters $n\in \mathbb{N}$
and $p \in [0,1]$:\[
\Pr(X = k) = {n\choose k}p^k(1-p)^{n-k}\!\quad \text{for }k\in\{0,\ldots,n\}.
\]

The Bernoulli distribution can be expressed as a special case of
the Binomial when $n=1$:\[
\Pr(X=k) = p^k (1-p)^{1-k}\!\quad \text{for }k\in\{0,1\}.
\]

A Binomial random variable is the sum of $n$ independent 0/1 Bernoulli
random variables, so it has $n$ times the expected value and the
variance: \[
\mbox{E}(X)=np
\] and variance \[
\mbox{Var}(X)=np(1-p)
\]
\end{frame}

\begin{frame}\frametitle{Bernoulli, Binomial, \textcolor{red}{Beta}}
The Beta distribution is a continuous probability distribution defined
on the interval $[0,1]$ with parameters $\alpha > 0$ and $\beta >
0$:\[
f(x;\alpha,\beta) = \frac{1}{\mbox{B}(\alpha,\beta)}\,
x^{\alpha-1}(1-x)^{\beta-1}
\]\[
E(X) = \frac{\alpha}{\alpha+\beta}\!\quad
\mbox{Mode\footnote{
Mode only exists if $\alpha,\beta>1$.
}}(X) = \frac{\alpha-1}{\alpha+\beta-2}
\]
The normalization constant $\mbox{B}(\alpha,\beta)$ is known as the
Beta function:\[
\mbox{B}(\alpha,\beta) = \int_0^1 x^{\alpha-1}(1-x)^{\beta-1} dx = \frac{\Gamma(\alpha)\Gamma(\beta)}{\Gamma(\alpha+\beta)}
\]
$\Gamma(\alpha)$ is known as the Gamma function and is a continuous extension of
the factorial function:\[
\Gamma(n) = (n-1)!
\]
\end{frame}

\begin{frame}\frametitle{Bernoulli, Binomial, Beta}
The Beta distribution is the {\em conjugate prior distribution} for the
Binomial distribution (i.e. when used as a prior distribution for the
$p$ parameter, the posterior for $p$ also has a Beta distribution).
\begin{itemize}
\item Parameter $p$ picked from a Beta distribution (prior):\[
f(p) \propto p^{\alpha-1}(1-p)^{\beta-1}
\]
\item Observe $k$ successes in $n$ independent Bernoulli trials (likelihood):\[
\mathcal{L}(p|k) = \Pr(k|p) \propto p^k(1-p)^{n-k}
\]
\item Update your belief about the value of $p$ (posterior):\[
f(p|k) = \frac{\Pr(k|p) f(p)}{\int_0^1\Pr(k|q)f(q)dq} \propto p^{k+\alpha-1}(1-p)^{n-k+\beta-1}
\]
\end{itemize}
\end{frame}

\begin{frame}\frametitle{Laplace - Rule of Succession}
The 18th century mathematician Pierre-Simon Laplace considered the
problem \begin{quote}
``What is the probability that the sun will rise tomorrow?''
\end{quote}
More generally if we observe $k$ successes out of $n$ independent
Bernoulli trials, what is the probability that the next trial will
succeed?
\end{frame}

\begin{frame}\frametitle{Laplace - Rule of Succession}
\begin{itemize}
\item Let us assume an uninformative prior for $p$ (a Beta
  distribution with $\alpha=\beta=1$):
$f(p) = 1 \!\quad \text{for }p\in\{0,1\}$
\item This will give us a posterior proportional to likelihood: \[
f(p|k) = \frac{p^k(1-p)^{n-k}}{\mbox{B}(k+1,n-k+1)}
\]
\item To predict what is going to happen next, we should consider all
  possible $p$ values (predictive distribution): \begin{eqnarray*}
\Pr(\mbox{success}|k) &=& \int_0^1 \Pr(\mbox{success}|p,k) f(p|k) dp \\
&=& \int_0^1 p\, f(p|k) dp = E_{p|k}(p)\\
&=& \frac{\int_0^1 p^{k+1}(1-p)^{n-k} dp}{\mbox{B}(k+1,n-k+1)} \\
&=& \frac{\mbox{B}(k+2,n-k+1)}{\mbox{B}(k+1,n-k+1)} = \frac{k+1}{n+2}\\
\end{eqnarray*}
\end{itemize}
\end{frame}

\begin{frame}\frametitle{Categorical, Multinomial, Dirichlet}
Generalize from 2 possible outcomes to $K$ possible outcomes:
\begin{itemize}
\item Bernoulli $\rightarrow$ Categorical distribution\footnote{
  alternatively $f(x|\mathbf{p})=\prod_{k=1}^K p_k^{[x=k]}$ or
  $f(\mathbf{x}|\mathbf{p})=\prod_{k=1}^K p_k^{x_k}$} 
\\(single trial, observed value $\in \{1,\dots, K\}$): \[ 
\Pr(X=k| \mathbf{p} ) = p_k\!\quad \text{for }k\in\{1,\ldots,K\}
\]

\item Binomial $\rightarrow$ Multinomial distribution\footnote{
$k\in\{1,\dots,K\}, x_k\in\{0,\dots,n\}, \sum x_k=n, p_k\in[0,1], \sum
  p_k = 1$.
}
\\($n$ trials, $x_k$ is the count of observing k):\[
\Pr(X_1 = x_1, \dots, X_k = x_k) = 
\frac{n!}{x_1!\cdots x_k!}\,p_1^{x_1}\cdots p_k^{x_k}
\]

\item Beta $\rightarrow$ Dirichlet distribution\footnote{$
\alpha_k>0, p_k\in[0,1], \sum p_k=1, 
\mbox{B}(\boldsymbol{\alpha}) = 
\left(\prod \Gamma(\alpha_k)\right)/\Gamma\left(\sum \alpha_k\right)
$}
(conjugate prior for $\mathbf{p}$):\[
f(\mathbf{p};\boldsymbol{\alpha}) = \frac{1}{\mbox{B}(\boldsymbol{\alpha})}\,
\prod_{k=1}^K p_k^{\alpha_k-1}
\]
\end{itemize}
\end{frame}

\begin{frame}\frametitle{Categorical, Multinomial, Dirichlet}
Some properties of the Dirichlet distribution:
\begin{itemize}
\item Define $\alpha = \sum \alpha_k$ and $m_k = \alpha_k / \alpha$.
  The vector $\mathbf{m}$ gives the mean and $\alpha$ measures
  sharpness (large $\alpha$ means distribution sharply peaked around
  $\mathbf{m}$): \[ f(\mathbf{p};\alpha \mathbf{m}) =
  \frac{1}{\mbox{B}(\alpha \mathbf{m})}\, \prod_{k=1}^K p_k^{\alpha
    m_k-1}
\]
\item Expectation: $E(p_k) = \frac{\alpha_k}{\alpha}$.

\item Mode: $p_k = \frac{\alpha_k - 1}{\alpha - K}$.

\item Marginals: $p_k \sim \mbox{Beta}(\alpha_k, \alpha - \alpha_k)$.

\item Aggregation: $(p_1,p_2,\ldots,p_i+p_j,\ldots,p_K) \sim 
\mbox{Dir}(\alpha_1,\alpha_2,\ldots,\alpha_i+\alpha_j,\ldots,\alpha_K)$.
\end{itemize}
\end{frame}

\begin{frame}\frametitle{Derivation of Additive (Laplace) Smoothing}
\begin{itemize}
\item Let us assume a symmetric Dirichlet prior for $\mathbf{p}$ with
  $\alpha_k = \alpha$:\[
f(\mathbf{p}) = \frac{1}{\mbox{B}(\boldsymbol{\alpha})}\,
\prod_{k=1}^K p_k^{\alpha-1}
\]

\item This will give us the following posterior: \[
f(\mathbf{p}|\mathbf{x}) \propto \prod p_k^{x_k + \alpha - 1}
\]

\item The mean of the posterior gives additive smoothing:\[
E(p_k) = \frac{x_k + \alpha}{\sum x_k + K\alpha}
\]

\item Is this the same as using the predictive distribution?

\item Why does it not work very well?
\end{itemize}
\end{frame}

%% \begin{frame}\frametitle{MacKay-Peto and Jelinek-Mercer}

%% \end{frame}

%% \begin{frame}\frametitle{Dirichlet Process, Pitman-Yor Processes}

%% \end{frame}

%% \begin{frame}\frametitle{Teh and Kneser-Ney}

%% \end{frame}

\end{document}
